\documentclass{article}
\title{CL-BibTeX example file}
\begin{document}
\maketitle

In this document, we cite three papers with keys \verb|Gom58| \cite{Gom58},
\verb|deyRichard| \cite{deyRichard}, and \verb|Lenstra1983| \cite{Lenstra1982}.
\medbreak

We illustrate two additional features of CL-BibTeX.

\begin{enumerate}
\item
There are actually two entries \verb|deyRichard| \cite{deyRichard} and
\verb|dey3| \cite{dey3} for the same article, and both are cited.  This often
happens in collaborations. BibTeX cannot recognize that both are the same
citation.  Neither can CL-BibTeX.  However, it provides a helpful facility for
this situation. Bib entries can be marked using the field
\verb|Equivalent-Entries|.  CL-BibTeX computes the equivalence classes
generated by these lists of equivalent entries.  If equivalent entries are
cited in one document (like in this example document), CL-BibTeX will emit a
warning.  The user can now fix the citations in the document.
\item
The bib entry for \verb|Lenstra1982| is slightly incorrect because Lenstra's
initials are not separated by a space.  In plain BibTeX, all abbrv bibtex
styles will abbreviate ``Lenstra, H.W.'' as ``Lenstra, H.''.  CL-BibTeX
corrects this and also outputs a warning, so that the user can fix this entry.
\end{enumerate}

\bibliography{example}
\bibliographystyle{amsabbrv}
\end{document}

%%% Local Variables:
%%% mode: latex
%%% TeX-master: t
%%% End:
